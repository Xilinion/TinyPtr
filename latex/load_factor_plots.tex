\section{Load Factor Support Analysis}

\begin{figure}[p]
    \centering
    \begin{tikzpicture}
        \begin{axis}[
            width=12cm,
            height=8cm,
            xlabel={Bin Size},
            ylabel={Load Factor Support (\%)},
            xmode=log,
            xmin=1,
            xmax=200,
            ymin=0,
            ymax=100,
            grid=major,
            grid style={gray!30},
            legend pos=outer north east,
            legend style={font=\small},
            tick label style={font=\small},
            label style={font=\small},
            title={Load Factor Support vs Bin Size},
            title style={font=\small},
            scaled ticks=true,
            tick label style={/pgf/number format/fixed,/pgf/number format/precision=1}
        ]

        % Plot insertion data (from CSV)
        \addplot[
            mark=o,
            mark size=3pt,
            color=exp1,
            thick,
            smooth
        ] table[
            x=bin_size,
            y index=3,
            col sep=comma,
            restrict expr to domain={\thisrow{object_id}}{4:4}
        ] {\loadfactordata};
        \addlegendentry{Insertion Only}

        % Plot deletion data (hardcoded)
        \addplot[
            mark=square,
            mark size=3pt,
            color=exp2,
            thick,
            smooth
        ] coordinates {
            (3,7)
            (7,43)
            (15,71)
            (31,85)
            (63,91)
            (127,95)
        };
        \addlegendentry{With Deletion}

        \end{axis}
    \end{tikzpicture}
    \caption{Load Factor Support vs Bin Size for different hash table implementations. The x-axis shows bin size on a logarithmic scale, while the y-axis shows the maximum supported load factor as a percentage.}
    \label{fig:load_factor_support}
\end{figure} 